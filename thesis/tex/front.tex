%
%  front.tex   2003.03.13  15:58:44   Mark D Senn <mds@purdue.edu>
%
%  This is the ``front matter'' for the example thesis.
%
%  Regarding ``References'' below:
%      KEY    MEANING
%      PU     ``A Manual for the Preparation of Graduate Theses'',
%             The Graduate School, Purdue University, 1996.
%      TCMOS  The Chicago Manual of Style, Edition 14.
%      WNNCD  Webster's Ninth New Collegiate Dictionary.
%
%  HISTORY:
%    2003.03.13 Added HISTORY section.
%               Uncommented "\listoftables" and "\lisoffigures".
%

  % Title page is required.
  % The title page is constructed using the information specified
  % with \title, \author, \degree\ and \majorprof earlier.
\maketitle

  % Dedication page is optional.
  % A name and often a message in tribute to a person or cause.
  % References: PU 15, WNNCD 332.
\begin{dedication}
  This work is dedicated to my greatest teachers; my parents.
\end{dedication}

  % Acknowledgements page is optional but most theses include
  % a brief statement of appreciation or recognition of special
  % assistance.
  % Reference: PU 16.
\begin{acknowledgments}
[Put statement of appreciation or recognition of special
assistance here.]
%I would first like to acknowledge my great appreciation to my
%major professor, Dan Elliott for giving my the opportunity to work
%on this project, and also for the numerous other opportunities he
%has given me to help enrich my education.

%I would like to thank the professors who agreed to serve on my
%advisory committee, Mark Bell and Andrew Weiner.  The contribution
%of their time is much appreciated.





\end{acknowledgments}

  % The preface is optional.
  % References: PU 16, TCMOS 1.49, WNNCD 927.
\begin{preface}
%  [Put introductory remarks here regarding reasons for undertaking this
%  work and method of research.
%
%  Since everyone knows you're writing a thesis to get your degree,
%  don't put that here.
%  If your research was done to solve a problem that came up in industry
%  you may want to put that here.
%
%  If not obvious from the rest or your thesis
%  you may want to describe your method of research here.
%
%  Acknowledgements should go in the ``Acknowledgments'' section
%  and don't belong here.]

In recent years, there has been growing interest in the field
known as "Coherent Control."   The continuing development of more
stable, more powerful lasers has made this even more possible.
Coherent control is a branch of science that studies the
possibility of using known properties of coherent light, such as
relative phase, to control various interactions. Applications
range from controlling photo-ionization rate~\cite{Schumacher,
Schumacher1} to creating directional currents in
semiconductors~\cite{Dupont, vanDriel} to making measurements of
atomic parameters~\cite{Wang3} and controlling the products in
photo-dissociation~\cite{Yin, Sheehy}.


Many coherent control experiments will exploit the coherence
properties between an optical field and its second harmonic.  In a
great deal of these, it is necessary to know the relative optical
phase between these two fields at some point in space; typically
where some interaction is taking place.  In general, relative
optical phase between two coherent beams of a single frequency can
be easily determined by observing the interference between the two
beams.  A method of doing this with a field and its second
harmonic was developed by Chudinov et al.~\cite{Chudinov}.
However, in order to adapt this for use in some of these coherent
control experiments, some additional considerations must be made.



\end{preface}

  % The Table of Contents is required.
  % The Table of Contents will be automatically created for you
  % using information you supply in
  %     \chapter
  %     \section
  %     \subsection
  %     \subsubsection
  % commands.
  % Reference: PU 16.
\tableofcontents

  % If your thesis has tables, a list of tables is required.
  % The List of Tables will be automatically created for you using
  % information you supply in
  %     \begin{table} ... \end{table}
  % environments.
  % Reference: PU 16.
\listoftables

  % If your thesis has figures, a list of figures is required.
  % The List of Figures will be automatically created for you using
  % information you supply in
  %     \begin{figure} ... \end{figure}
  % environments.
  % Reference: PU 16.
\listoffigures

  % List of Symbols is optional.
  % Reference: PU 17.
%\begin{symbols}
%  $m$& mass\cr
%  $v$& velocity\cr
%\end{symbols}

  % List of Abbreviations is optional.
  % Reference: PU 17.
%\begin{abbreviations}
%  abbr& abbreviation\cr
%  bcf& billion cubic feet\cr
%  BMOC& big man on campus\cr
%\end{abbreviations}

  % Nomenclature is optional.
  % Reference: PU 17.
%\begin{nomenclature}
%  Alanine& 2-Aminopropanoic acid\cr
%  Valine& 2-Amino-3-methylbutanoic acid\cr
%\end{nomenclature}

  % Glossary is optional
  % Reference: PU 17.
%\begin{glossary}
%  chick& female, usually young\cr
%  dude& male, usually young\cr
%\end{glossary}

  % Abstract is required.
  % Note that the information for the first paragraph of the output
  % doesn't need to be input here...it is put in automatically from
  % information you supplied earlier using \title, \author, \degree,
  % and \majorprof.
  % Reference: PU 17.
\begin{abstract}
%  State the problem, mention the research, and summarize the results
%  in 25 or fewer lines on the output.

We require knowledge of the relative phase between a field and its
second harmonic at a certain point in space for use in a coherent
control experiment. We will determine a method for measuring this
phase difference in focused Gaussian beams.  We will discuss the
importance of various factors that have been overlooked in our
previous experiments.

In order to make the relative phase measurement, a measurement
will be made using a $\beta$-Barium Borate frequency doubling
crystal that will allow us to determine the phase shift that
occurs between a fundamental beam and its second harmonic during
second harmonic generation.  With this knowledge, and using
various properties of focused Gaussian beams, we will be able to
determine a scheme for finding the desired optical phase
difference.

\end{abstract}
